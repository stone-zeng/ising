机器学习在计算机科学界无疑是发展最为迅速的领域之一。随着研究的深入,机器学习在物理学中的应用也逐渐
为人们所重视。然而,除了利用机器的强大计算能力进行数据处理,深入探讨机器学习中隐含的物理背景,
显然更为本质和重要。

作为统计物理中最为基本的模型,Ising 模型具有十分丰富的内涵。二维严格解的存在,则为各种研究提供了
准确的检验标准。Ising 模型在共形场论中的重要地位,又可以与 \AdSCFT{} 相联系,从而得到更深层次的
物理。另一方面,成熟的数值模拟技术(如 Monte Carlo 等),为提供 Ising 模型的大量数据做了保障。据此,
在大规模训练数据集的基础上,运用机器学习手段获得相应的物理结果,也就成为了可能。

本文我们将以 Ising 模型作为主要研究对象,在其基础上引入多种机器学习手段,进而探讨有关的物理图像。

第 1 章里,我们将简要概述 Ising 模型的统计物理,并基于重整化群讨论其临界现象。我们还将简单介绍
Monte Carlo 算法的原理及有关模拟结果。

在第 2 章,我们将介绍机器学习的基本思想方法。作为例子,我们分别利用线性回归模型和多层人工神经网络
对 Ising 模型数据进行学习,展示其对各相进行分类的能力。

第 3 章是本文的重点。首先我们将给出限制 Boltzmann 机的网络结构,推导对比散度算法,并在 MNIST
手写数字数据集上对其进行训练。接着,我们将探讨限制 Boltzmann 机与重整化群的对应关系,并对 Ising
模型数据上的训练结果予以讨论。最后引入卷积层,考虑卷积与小波变换和 \AdSCFT{} 之间的联系,从理论上
分析利用卷积限制 Boltzmann 机获取定量结果的可能性。

第 4 章给出了本文的主要结论。
