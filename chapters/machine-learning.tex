\chapter{机器学习简介}

\section{机器学习有关思想方法}

机器学习是当今计算机科学的重要分支。它可以利用计算机的强大计算能力,从大量数据中提取特征,并构建相应的模型。生成的模型还可以用来进行预测。

考虑最简单的回归模型。假设我们给定了一些数据点 $(\bm{x}^{(i)}, \, y^{(i)})$,其中 $\bm{x}^{(i)}$ 称为特征(feature),$y^{(i)}\in\R$ 称为标签(label)。训练样本的数量为 $m$,即 $i=1,\,2,\,\ldots,\,m$。特征的数量为 $n$,但按照惯例,取 $x_0=0$,故有 $\bm{x}^{(i)}\in\R^{n+1}$。一般而言,特征反映了样本的各种性质(如图片中的像素点),而标签则是需要需要预测的内容(如图片对应的物体类别)。为了学习样本数据,我们需要给出一个目标函数 $h_\bm{\theta}: \R^{n+1}\to\R$,使得 $h_\bm{\theta}(\hat{\bm{x}})=\hat{y}$。这里的 $\bm{\theta}$ 称为参数(parameter),它包含了模型的全部训练信息。

显然,如果要求学习的效果越好,就要求预测值 $\hat{y}$ 与真实值 $y$ 的差距越小。因此我们需要定义损失函数(loss function) $L$,并使 $L$ 关于参数 $\bm{\theta}$ 取最小值。如果数据规模较小、模型比较简单,$L$ 的最小值可以解析求得。例如线性回归模型,即
\begin{equation}
  h_\bm{\theta} (\bm{x})
  = \theta_0 + \theta_1 x_1 + \theta_2 x_2 + \cdots + \theta_n x_n
  = \sum_{j=0}^n \theta_j x_j,
\end{equation}
我们取损失函数为
\begin{equation}
  L(\bm{\theta})
  = \frac{1}{2} \sum_{i=1}^m \qty\Big[y^{(i)} - h_\bm{\theta}\qty\big(\bm{x}^{(i)})]^2
  = \frac{1}{2} \norm{\bm{y} - \bm{X}\bm{\theta}}^2 \in \R.
\end{equation}
式中 $\bm{X}\in\R^{m\times(n+1)}$,且有 $X_{ij} = x^{(i)}_j$。利用最小二乘估计,可以求得最优解
\begin{equation}
  \hat{\bm{\theta}} = \qty\big(\bm{X}^\trans\bm{X})^{-1} \bm{X}^\trans \bm{y}.
\end{equation}

然而,当数据规模很大时,可以发现矩阵 $\bm{X}^\trans\bm{X}$ 的阶数也会很大,导致求其逆阵的开销无法接受。因此可以改换思路,利用梯度下降法逼近损失函数的最小值。对于一般的多元函数,梯度给出了其在某一点处变化率最大的方向。因而沿着梯度方向不断改变 $\bm{\theta}$ 的值,就可以逐步趋近最小值。我们可以令
\begin{equation}
  \bm{\theta} \coloneq \bm{\theta} - \alpha\nabla\bm{\theta},
\end{equation}
并不断循环,直至收敛。式中 $\alpha$ 称为学习率,用来调节每次下降的“步幅”。

一般来说,机器学习的思路可以概括如下:

\begin{enumerate}
  \item 根据数据的具体类型、特性构建相应的模型;
  \item 给出模型的损失函数;
  \item 在数据集上进行训练,找出能使损失函数取最小值的参数;
  \item 验证模型及参数的有效性,进行参数微调。
  \item 在新的数据集上测试训练好的模型。
\end{enumerate}

\section{对 Ising 模型相变的研究}
