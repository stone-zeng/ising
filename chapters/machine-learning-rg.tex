\chapter{机器学习与重整化群}

\section{限制 Boltzmann 机(RBM)}

\subsection{模型构建}

限制 Boltzmann 机(restricted Boltzmann machine, RBM)是一类生成型随机神经网络
(generative stochastic artificial neural network),它可以学习到输入数据的概率分布情况。

Boltzmann 机是一种基于能量的模型(energy-based model)。对于这种模型,我们可以赋予它一个能量函数。
与物理学中类似,体系的稳定状态将在能量最低时达到,这一稳定状态即对应最优参数。

Boltzmann 机的结构如图~\ref{fig:boltzmann-machine} 所示。它共有两层,与数据直接相连的称为可见层
(visible layer),另一层称为隐藏层(hidden layer)。可见层用于处理输入输出,隐藏层则反应了数据的
内在结构。无论是可见层还是隐藏层,其中的所有单元均全部连接在一起。Boltzmann 机的能量函数由下式给出:
\begin{equation}
  E(\bm{s}) = E(\q{s_i}) = -\sum_{i<j} W_{ij} s_i s_j - \sum_i \theta_i s_i.
\end{equation}
式中,$s_i$ 取为布尔型,即 $s_i\in\q{0,\,1}$。$W_{ij}$ 是单元 $s_i$ 与 $s_j$ 之间的连接权重,
$\theta_i$ 则表示单元 $s_i$ 处的偏差。

\begin{figure}[htb]
  \centering
  \begin{subfigure}[b]{0.45\textwidth}
    \centering
    \tikzinput{boltzmann-machine}
    \caption{Boltzmann 机}
    \label{fig:boltzmann-machine}
  \end{subfigure}
  \begin{subfigure}[b]{0.45\textwidth}
    \centering
    \tikzinput{rbm}
    \vspace{0.4cm}
    \caption{限制 Boltzmann 机}
    \label{fig:rbm}
  \end{subfigure}
  \caption{Boltzmann 机和限制 Boltzmann 机的结构示意图。其中红色圆圈表示可见层,蓝色方块表示隐藏层。
  Boltzmann 机的所有单元之间均有权重矩阵连接;而在限制 Boltzmann 机中,只有可见层与隐藏层之间才连接
  有权重矩阵}
\end{figure}

与 Ising 模型类似[式~\eqref{eq:ising-probability}],状态 $\bm{s}=\q{s_i}$ 出现的概率由 Boltzmann
分布给出:
\begin{equation}
  \label{eq:boltzmann-machine-probability}
  P(\bm{s}) = \frac{\ee^{-E(\bm{s})}}{Z},
\end{equation}
其中配分函数 $Z$ 的定义为
\begin{equation}
  Z = \sum_{\bm{s}} \ee^{-E(\bm{s})}.
\end{equation}
Boltzmann 机的训练过程就是最大化训练样本所对应的概率。

由于 Boltzmann 机是全连接的,它的训练效率并不高。我们可以引入限制条件,仅保留可见层与隐藏层之间的
连接。由此便得到了限制 Boltzmann 机(如图~\ref{fig:rbm}),它的能量函数为
\begin{align}
  E(\bm{v},\,\bm{h})
  &= - \bm{v}^\trans \bm{W} \bm{h} - \bm{b}^\trans \bm{v} - \bm{c}^\trans \bm{h} \notag \\
  &= - \sum_{i,\,j} W_{ij} v_i h_j - \sum_i b_i v_i - \sum_j c_j h_j.
\end{align}
式中,$v_i$ 和 $h_j$ 分别代表可见层与隐藏层中的单元,$W_{ij}$ 为两层之间的连接权重,$b_i$ 和 $c_j$
分别是可见层与隐藏层中单元对应的偏差。

限制 Boltzmann 机中,可见层 $\bm{v}$ 与隐藏层 $\bm{h}$ 是彼此分离的,可以视为两个随机变量,因而
式~\eqref{eq:boltzmann-machine-probability} 即成为 $\bm{v}$ 与 $\bm{h}$ 的联合概率分布:
\begin{equation}
  P(\bm{v},\,\bm{h})
  = \frac{\ee^{-E(\bm{v},\,\bm{h})}}{Z}
  = \frac{\ee^{-E(\bm{v},\,\bm{h})}}{\sum_{\bm{v},\,\bm{h}}\ee^{-E(\bm{v},\,\bm{h})}}.
\end{equation}
对所有的 $\bm{h}$ 进行求和,我们就得到了某一可见层分布 $\bm{v}$ 对应的概率:
\begin{equation}
  P(\bm{v}) = \frac{1}{Z} \sum_{\bm{h}} P(\bm{v},\,\bm{h})
            = \frac{1}{Z} \sum_{\bm{h}} \ee^{-E(\bm{v},\,\bm{h})}.
\end{equation}

\subsection{训练算法}


\section{RBM 与重整化群的对应关系}
\section{利用 RBM 研究 Ising 模型}
\section{卷积限制 Boltzmann 机(CRBM)}
\section{卷积层、小波变换与 \AdSCFT{}}
\section{利用 CRBM 研究 Ising 模型}
