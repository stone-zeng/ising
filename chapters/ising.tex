\chapter{Ising 模型}

\section{Ising 模型的统计物理}

Ising 模型是统计物理中描述铁磁系统的一种简单模型。Ising 模型由一系列自旋排列而成的点阵构成,其中的每
个自旋只有两种取向,即 $+1$ 或 $-1$。自旋与自旋之间的相互作用可以导致相变的出现,最典型的例子是二维
正方晶格的 Ising 模型,其严格解由 Onsager 给出。

Ising 模型的 Hamilton 量由下式给出:
\begin{equation}
  H(\q{\sigma_i}) = -\sum_{\nearest{ij}} J_{ij}\sigma_i\sigma_j - \mu \sum_{i} B_i\sigma_i.
\end{equation}
式中,“$\nearest{ij}$” 表示相邻的自旋。$J_{ij}$ 是耦合常数,通常情况下与自旋位置无关,即
$J_{ij}=J$,因而可以放在求和号外部。$J>0$ 时,代表体系处于铁磁态;$J<0$ 时,则代表处于反铁磁态。
$B_i$ 是外磁场,对于匀强磁场,有 $B_i=B$,即同样与位置无关。$\mu$ 是对应的磁矩。由于体系关于
$\sigma_i$ 的正负是对称的,方便起见不妨认为 $B \geqslant 0$。此时,我们有
\begin{equation}
  H(\q{\sigma_i}) = -J \sum_{\nearest{ij}} \sigma_i\sigma_j - \mu B \sum_{i} \sigma_i.
\end{equation}

我们可以为某一特定的自旋构型引入概率。温度 $T$ 时,构型 $\q{\sigma_i}$ 在平衡态中出现的概率由
Boltzmann 分布给出:
\begin{equation}
  P(\q{\sigma_i}) = \frac{\ee^{-\beta H(\q{\sigma_i})}}{Z_N}.
\end{equation}
其中的 $\beta=1/\kB T$。而 $Z$ 称为配分函数,它使得概率 $P$ 满足归一化条件
\begin{equation}
  \sum_\q{\sigma_i} P(\q{\sigma_i}) = 1
  \implies Z_N = \sum_\q{\sigma_i} \ee^{-\beta H(\q{\sigma_i})}.
\end{equation}
配分函数等价于体系的自由能。在温度 $T$、外磁场 $B$ 作为变量的情况下,自由能是体系的特性函数,故可以
通过自由能计算出各物理量,包括内能、热容、磁化强度等。因此求解 Ising 模型实际上就是求出体系的配分函
数(自由能)。

一维 Ising 模型的严格解可利用转移矩阵的手段求出。热力学极限下(自旋数目 $N\to\infty$),自由能为
\begin{equation}
  F = - \frac{\ln Z_N}{\beta}
    = - N J - \frac{N}{\beta}
              \ln\qty[\cosh(\beta\mu B) + \sqrt{\ee^{-4\beta J} + \sinh^2(\beta\mu B)}].
\end{equation}
磁化强度为
\begin{equation}
  \frac{\bar{M}}{N\mu} = \frac{\sinh{\beta\mu B}}{\sqrt{\ee^{-4\beta J} + \sinh^2(\beta\mu B)}}.
\end{equation}
显然,只要 $\beta<\infty$,当 $B\to 0$ 时,$\bar{M}\to 0$;而当 $\beta\to 0$ 时,$\bar{M}\to N\mu$。
这说明一维 Ising 模型的临界温度 $\Tc=0$,事实上不存在有限温度下的相变。

二维 Ising 模型的严格解比较复杂。对于正方晶格,可利用对偶变换求出临界温度
\begin{equation}
  \frac{\Tc}{J} = \frac{2}{\ln(1+\sqrt{2})} \approx \num{2.2692}.
\end{equation}
外场为零时,可以求得(单位自旋的)自由能为
\begin{equation}
  \frac{\beta F}{N}
  = - \frac{\ln Z_N}{N}
  = - \frac{\ln 2}{2} - \ln\cosh 2\beta J
    - \frac{1}{2\pp} \int_0^\pp \ln\qty[1+\sqrt{1-\kappa^2\sin^2\theta}] \dd{\theta},
\end{equation}
其中
\begin{equation}
  \kappa = \frac{2\sinh 2\beta J}{\cosh^2 2\beta J}.
\end{equation}
进而可以求得热容:
\begin{align}
  \frac{C}{N}
  & \simeq -\frac{2}{\pp} \ln^2 \qty(1+\sqrt{2}) \ln\abs{1-\frac{T}{\Tc}} + \const \notag \\
  & \simeq \num{-0.4945} \ln\abs{1-\frac{T}{\Tc}} + \const.
\end{align}
可以看到热容在临界温度处呈现出对数发散行为。

磁化强度的计算结果由杨振宁首次给出:
\begin{equation}
  \frac{\bar{M}}{N\mu} =
  \begin{cases}
    \qty(1 - \sinh^{-4} 2\beta J)^{1/8}, & T<\Tc; \\
    0, & T>\Tc.
  \end{cases}
\end{equation}

\section{临界指数}

\section{重整化群}
\section{Ising 模型与 \AdSCFT{}}
\section{Ising 模型的 Monte Carlo 模拟}
